\documentclass[11pt, reqno]{amsart}
\usepackage{amsfonts, amssymb, amscd, amsrefs}
\usepackage{graphicx}
\usepackage{hyperref}
\usepackage{slashed}
\usepackage{fullpage}
% Prevent table repositioning.
\usepackage{float}
% For textcolor.
\usepackage{xcolor}
% For blackboard bold `1`.
\usepackage{bbold}

% TODO(gp): It aborts with
% ! Undefined control sequence.
%<argument> ...on\endcsname \protect \@secnumpunct
%
%l.164 \subsection{Breaking the symmetry}
% https://tex.stackexchange.com/questions/165930/bold-and-italic-subsection-title-with-custom-font-size
%\usepackage{titlesec}
%\titleformat{\section}
%{\normalfont\fontfamily{phv}\fontsize{12}{17}\bfseries}{\thesection}{1em}{}
%\titleformat{\subsection}
%{\normalfont\fontfamily{phv}\fontsize{12}{17}\bfseries\itshape}{\thesubsection}{1em}{}
%\titleformat{\subsection}
%  {\normalfont\fontsize{12}{17}\sffamily\bfseries\slshape}
%  {\thesubsection}
%  {1em}
%  {}

%\usepackage{titlesec}

%\titleformat*{\section}{\LARGE\bfseries}
%\titleformat*{\subsection}{\Large\bfseries}
%\titleformat*{\subsubsection}{\large\bfseries}
%\titleformat*{\paragraph}{\large\bfseries}
%\titleformat*{\subparagraph}{\large\bfseries}

% https://ctan.math.washington.edu/tex-archive/macros/latex/required/amscls/doc/amsthdoc.pdf
\newtheorem{thm}{Theorem}
\theoremstyle{definition}
\newtheorem{defn}{Definition}[subsection]
\newtheorem{problem}{Problem}[subsection]
\theoremstyle{remark}
\newtheorem{exmp}{Example}[subsection]
\newtheorem{rmk}{Remark}[subsection]

\newcommand{\bidbtc}{\mathrm{bid}_\mathrm{BTC}}
\newcommand{\askbtc}{\mathrm{ask}_\mathrm{BTC}}
\newcommand{\bideth}{\mathrm{bid}_\mathrm{ETH}}
\newcommand{\asketh}{\mathrm{ask}_\mathrm{ETH}}

\newcommand{\bidbase}{\mathrm{bid}_\mathrm{quote\;token}}
\newcommand{\askbase}{\mathrm{ask}_\mathrm{quote\;token}}
\newcommand{\bidquote}{\mathrm{bid}_\mathrm{base\;token}}
\newcommand{\askquote}{\mathrm{ask}_\mathrm{base\;token}}
% Use teletype for tokens (\texttt{...}), but do not allow italics
%  (\textnormal{...}).
\newcommand{\BTC}{\textnormal{\texttt{wBTC}}}
\newcommand{\ETH}{\textnormal{\texttt{ETH}}}
\newcommand{\DAI}{\textnormal{\texttt{DAI}}}
\newcommand{\USDC}{\textnormal{\texttt{USDC}}}
\newcommand{\USDT}{\textnormal{\texttt{USDT}}}
\newcommand{\tA}{\textnormal{\texttt{A}}}
\newcommand{\tB}{\textnormal{\texttt{B}}}
% Use non-italic teletype for order attributes in order notation.
\newcommand{\timestamp}{\textnormal{\texttt{timestamp}}}
\newcommand{\action}{\textnormal{\texttt{action}}}
\newcommand{\quantity}{\textnormal{\texttt{quantity}}}
\newcommand{\basetoken}{\textnormal{\texttt{base\_token}}}
\newcommand{\limitprice}{\textnormal{\texttt{limit\_price}}}
\newcommand{\quotetoken}{\textnormal{\texttt{quote\_token}}}
\newcommand{\depositaddress}{\textnormal{\texttt{deposit\_address}}}
\newcommand{\zkpcomplexity}{\textnormal{\texttt{zkp\_complexity}}}
\newcommand{\zkpdeadline}{\textnormal{\texttt{zkp\_deadline}}}
%
\newcommand{\buy}{\textnormal{\texttt{buy}}}
\newcommand{\sell}{\textnormal{\texttt{sell}}}
\newcommand{\nan}{\textnormal{\texttt{nan}}}
%
\newcommand{\midpoint}{\mathrm{midpoint}}
% Table icons.
\newcommand{\trm}{\textcolor{red}{-}}
\newcommand{\tyo}{\textcolor{yellow}{o}}
\newcommand{\tgp}{\textcolor{green}{+}}

% From https://tex.stackexchange.com/questions/75667/change-colour-on-chapter-section-headings-lyx
\usepackage{xcolor}
%\usepackage{sectsty}
%\chapterfont{\color{blue}}
%\sectionfont{\color{red}}
%\subsectionfont{\color{purple}}
%\subsubsectionfont{\color{blue}}

% From https://tex.stackexchange.com/questions/1455/how-to-set-the-font-for-a-section-title-and-chapter-etc
%\usepackage{titlesec}
%\titleformat{\chapter}[display]
%  {\normalfont\sffamily\huge\bfseries\color{blue}}
%  {\chaptertitlename\ \thechapter}{20pt}{\Huge}
%\titleformat{\section}
%  {\normalfont\sffamily\Large\bfseries\color{cyan}}
%  {\thesection}{1em}{}

% Work-around for too deeply nested error.
\usepackage{enumitem}
\setlistdepth{9}
\setlist[itemize,1]{label=\textbullet}
\setlist[itemize,2]{label=\textendash}
\setlist[itemize,3]{label=\textasteriskcentered}
\setlist[itemize,4]{label=\textperiodcentered}
\setlist[itemize,5]{label=\textbullet}
\setlist[itemize,6]{label=\textbullet}
\setlist[itemize,7]{label=\textbullet}
\setlist[itemize,8]{label=\textbullet}
\setlist[itemize,9]{label=\textbullet}
\renewlist{itemize}{itemize}{9}

% #############################################################################
% Vector / matrix notation
% #############################################################################

% \vec is already defined so we use \vv

% Use bold.
%\newcommand{\vv}[1]{\boldsymbol{#1}}
%\newcommand{\mat}[1]{\boldsymbol{#1}}

% Use underline.
% From https://latex.org/forum/viewtopic.php?t=17798
% ulem doesn't work well in math mode.
%\newcommand{\vv}[1]{\boldsymbol{\uline{#1}}}
%\newcommand{\mat}[1]{\boldsymbol{\uuline{#1}}}
\newcommand{\vv}[1]{\boldsymbol{\underline{#1}}}
\newcommand{\mat}[1]{\boldsymbol{\underline{\underline{#1}}}}

% TODO(saggese): Remove this and use \quad.
\newcommand{\eqspace}{\hspace{10 mm}}
\newcommand{\norm}[1]{\left\lVert#1\right\rVert}

% NOTE: These shortcuts are not only to make the latex code easier but also to
% make easy to find out what the standard representation is.
% In general we prefer to put \vv{} and \mat{} at outermost.

% #############################################################################
% Newshortcuts for matrices and vectors.
% #############################################################################

% - v* for vectors, e.g., va, vb, ..., vmu, vA, vGamma
% - m* for matrices, e.g., mA, ..., mSigma

% Vector.
\newcommand{\vA}{\vv{A}}
\newcommand{\vB}{\vv{B}}
\newcommand{\vC}{\vv{C}}
\newcommand{\vD}{\vv{D}}
\newcommand{\vDelta}{\vv{\Delta}}
\newcommand{\vE}{\vv{E}}
\newcommand{\vF}{\vv{F}}
\newcommand{\vG}{\vv{G}}
\newcommand{\vGamma}{\vv{\Gamma}}
\newcommand{\vH}{\vv{H}}
\newcommand{\vI}{\vv{I}}
\newcommand{\vJ}{\vv{J}}
\newcommand{\vK}{\vv{K}}
\newcommand{\vL}{\vv{L}}
\newcommand{\vLambda}{\vv{\Lambda}}
\newcommand{\vM}{\vv{M}}
\newcommand{\vN}{\vv{N}}
\newcommand{\vO}{\vv{O}}
\newcommand{\vOmega}{\vv{\Omega}}
\newcommand{\vP}{\vv{P}}
\newcommand{\vPhi}{\vv{\Phi}}
\newcommand{\vPi}{\vv{\Pi}}
\newcommand{\vPsi}{\vv{\Psi}}
\newcommand{\vQ}{\vv{Q}}
\newcommand{\vR}{\vv{R}}
\newcommand{\vS}{\vv{S}}
\newcommand{\vSigma}{\vv{\Sigma}}
\newcommand{\vT}{\vv{T}}
\newcommand{\vU}{\vv{U}}
\newcommand{\vUpsilon}{\vv{\Upsilon}}
\newcommand{\vV}{\vv{V}}
\newcommand{\vW}{\vv{W}}
\newcommand{\vX}{\vv{X}}
\newcommand{\vXi}{\vv{\Xi}}
\newcommand{\vY}{\vv{Y}}
\newcommand{\vZ}{\vv{Z}}
\newcommand{\va}{\vv{a}}
\newcommand{\valpha}{\vv{\alpha}}
\newcommand{\vb}{\vv{b}}
\newcommand{\vbeta}{\vv{\beta}}
\newcommand{\vc}{\vv{c}}
\newcommand{\vchi}{\vv{\chi}}
\newcommand{\vd}{\vv{d}}
\newcommand{\vdelta}{\vv{\delta}}
\newcommand{\ve}{\vv{e}}
\newcommand{\vepsilon}{\vv{\epsilon}}
\newcommand{\veta}{\vv{\eta}}
\newcommand{\vf}{\vv{f}}
\newcommand{\vg}{\vv{g}}
\newcommand{\vgamma}{\vv{\gamma}}
\newcommand{\vh}{\vv{h}}
\newcommand{\vi}{\vv{i}}
\newcommand{\viota}{\vv{\iota}}
\newcommand{\vj}{\vv{j}}
\newcommand{\vk}{\vv{k}}
\newcommand{\vkappa}{\vv{\kappa}}
\newcommand{\vl}{\vv{l}}
\newcommand{\vlambda}{\vv{\lambda}}
\newcommand{\vm}{\vv{m}}
\newcommand{\vmu}{\vv{\mu}}
\newcommand{\vn}{\vv{n}}
\newcommand{\vnu}{\vv{\nu}}
\newcommand{\vo}{\vv{o}}
\newcommand{\vomega}{\vv{\omega}}
\newcommand{\vp}{\vv{p}}
\newcommand{\vphi}{\vv{\phi}}
\newcommand{\vpi}{\vv{\pi}}
\newcommand{\vpsi}{\vv{\psi}}
\newcommand{\vq}{\vv{q}}
\newcommand{\vr}{\vv{r}}
\newcommand{\vrho}{\vv{\rho}}
\newcommand{\vs}{\vv{s}}
\newcommand{\vsigma}{\vv{\sigma}}
\newcommand{\vt}{\vv{t}}
\newcommand{\vtau}{\vv{\tau}}
\newcommand{\vtheta}{\vv{\theta}}
\newcommand{\vu}{\vv{u}}
\newcommand{\vupsilon}{\vv{\upsilon}}
\newcommand{\vvarepsilon}{\vv{\varepsilon}}
\newcommand{\vvarphi}{\vv{\varphi}}
\newcommand{\vvarrho}{\vv{\varrho}}
\newcommand{\vvartheta}{\vv{\vartheta}}
\newcommand{\vvv}{\vv{v}}
\newcommand{\vw}{\vv{w}}
\newcommand{\vx}{\vv{x}}
\newcommand{\vxi}{\vv{\xi}}
\newcommand{\vy}{\vv{y}}
\newcommand{\vz}{\vv{z}}
\newcommand{\vzeta}{\vv{\zeta}}

% Matrix.
% \mat -> \mm?
\newcommand{\mA}{\mat{A}}
\newcommand{\mB}{\mat{B}}
\newcommand{\mC}{\mat{C}}
\newcommand{\mD}{\mat{D}}
\newcommand{\mDelta}{\mat{\Delta}}
\newcommand{\mE}{\mat{E}}
\newcommand{\mF}{\mat{F}}
\newcommand{\mG}{\mat{G}}
\newcommand{\mGamma}{\mat{\Gamma}}
\newcommand{\mH}{\mat{H}}
\newcommand{\mI}{\mat{I}}
\newcommand{\mJ}{\mat{J}}
\newcommand{\mK}{\mat{K}}
\newcommand{\mL}{\mat{L}}
\newcommand{\mLambda}{\mat{\Lambda}}
\newcommand{\mM}{\mat{M}}
\newcommand{\mN}{\mat{N}}
\newcommand{\mO}{\mat{O}}
\newcommand{\mOmega}{\mat{\Omega}}
\newcommand{\mP}{\mat{P}}
\newcommand{\mPi}{\mat{\Pi}}
\newcommand{\mPsi}{\mat{\Psi}}
\newcommand{\mQ}{\mat{Q}}
\newcommand{\mR}{\mat{R}}
\newcommand{\mS}{\mat{S}}
\newcommand{\mSigma}{\mat{\Sigma}}
\newcommand{\mT}{\mat{T}}
\newcommand{\mU}{\mat{U}}
\newcommand{\mUpsilon}{\mat{\Upsilon}}
\newcommand{\mV}{\mat{V}}
\newcommand{\mW}{\mat{W}}
\newcommand{\mX}{\mat{X}}
\newcommand{\mXi}{\mat{\Xi}}
\newcommand{\mY}{\mat{Y}}
\newcommand{\mZ}{\mat{Z}}

% #############################################################################
% Other shortcuts
% #############################################################################

% Number fields
% TODO: use \R, \N, ... and the standard symbols (see
% http://oeis.org/wiki/List_of_LaTeX_mathematical_symbols)
\newcommand{\bbB}{\mathbb{B}}
\newcommand{\bbC}{\mathbb{C}}
\newcommand{\bbF}{\mathbb{F}}
\newcommand{\bbL}{\mathbb{L}}
\newcommand{\bbN}{\mathbb{N}}
\newcommand{\bbQ}{\mathbb{Q}}
\newcommand{\bbR}{\mathbb{R}}
\newcommand{\bbS}{\mathbb{S}}
\newcommand{\bbZ}{\mathbb{Z}}

% Script vars.
\newcommand{\cala}{\mathcal{a}}
\newcommand{\calb}{\mathcal{b}}
\newcommand{\calc}{\mathcal{c}}
\newcommand{\cald}{\mathcal{d}}
\newcommand{\cale}{\mathcal{e}}
\newcommand{\calf}{\mathcal{f}}
\newcommand{\calg}{\mathcal{g}}
\newcommand{\calh}{\mathcal{h}}
\newcommand{\cali}{\mathcal{i}}
\newcommand{\calj}{\mathcal{j}}
\newcommand{\calk}{\mathcal{k}}
\newcommand{\call}{\mathcal{l}}
\newcommand{\calm}{\mathcal{m}}
\newcommand{\caln}{\mathcal{n}}
\newcommand{\calo}{\mathcal{o}}
\newcommand{\calp}{\mathcal{p}}
\newcommand{\calq}{\mathcal{q}}
\newcommand{\calr}{\mathcal{r}}
\newcommand{\cals}{\mathcal{s}}
\newcommand{\calt}{\mathcal{t}}
\newcommand{\calu}{\mathcal{u}}
\newcommand{\calv}{\mathcal{v}}
\newcommand{\calw}{\mathcal{w}}
\newcommand{\calx}{\mathcal{x}}
\newcommand{\caly}{\mathcal{y}}
\newcommand{\calz}{\mathcal{z}}
\newcommand{\calA}{\mathcal{A}}
\newcommand{\calB}{\mathcal{B}}
\newcommand{\calC}{\mathcal{C}}
\newcommand{\calD}{\mathcal{D}}
\newcommand{\calE}{\mathcal{E}}
\newcommand{\calF}{\mathcal{F}}
\newcommand{\calG}{\mathcal{G}}
\newcommand{\calH}{\mathcal{H}}
\newcommand{\calI}{\mathcal{I}}
\newcommand{\calJ}{\mathcal{J}}
\newcommand{\calK}{\mathcal{K}}
\newcommand{\calL}{\mathcal{L}}
\newcommand{\calM}{\mathcal{M}}
\newcommand{\calN}{\mathcal{N}}
\newcommand{\calO}{\mathcal{O}}
\newcommand{\calP}{\mathcal{P}}
\newcommand{\calQ}{\mathcal{Q}}
\newcommand{\calR}{\mathcal{R}}
\newcommand{\calS}{\mathcal{S}}
\newcommand{\calT}{\mathcal{T}}
\newcommand{\calU}{\mathcal{U}}
\newcommand{\calV}{\mathcal{V}}
\newcommand{\calW}{\mathcal{W}}
\newcommand{\calX}{\mathcal{X}}
\newcommand{\calY}{\mathcal{Y}}
\newcommand{\calZ}{\mathcal{Z}}

% tilde
% TODO(gp): Expand this.
\newcommand{\ttau}{\tilde{\tau}}
% Irregular to avoid conflict.
\newcommand{\ttt}{\tilde{t}}
\newcommand{\ty}{\tilde{y}}

% TODO(gp): Expand this.
\newcommand{\hF}{\widehat{F}}
\newcommand{\hvF}{\vv{\widehat{F}}}
\newcommand{\hG}{\widehat{G}}
\newcommand{\hvG}{\vv{\widehat{G}}}
\newcommand{\hX}{\widehat{X}}
\newcommand{\hvX}{\vv{\widehat{X}}}

% Unit vector.
% Same as the vector name with unit
% TODO(gp): Merge with above.
\newcommand{\vvvhat}{\vv{\hat{v}}}
\newcommand{\uuhat}{\vv{\hat{u}}}
\newcommand{\xxhat}{\vv{\hat{x}}}
%
\newcommand{\tildeAAA}{\mat{\tilde{A}}}

% Basic.

% Define.
\def\d={\overset{\begin{tiny}d\end{tiny}}{=}}
%\def\d={\triangleq}
% Which was defined as.
\def\dd={\overset{\begin{tiny}d. as\end{tiny}}{=}}
% TODO: remove this
\def\diff{\overset{\begin{tiny}d\end{tiny}}{\iff}}
% TODO: This is already defined, but not clear where
%\newcommand{\gcd}{\text{gcd}}
\newcommand{\lcm}{\text{lcm}}

% Mean and variance.
\newcommand{\EE}{\mathbb{E}}
\newcommand{\VV}{\mathbb{V}}
\newcommand{\Cov}{\text{Cov}}
\newcommand{\Corr}{\text{Corr}}
%\newcommand{\Pr}{\mathbb{P}}

% Distributions.
% text{} or textsf
\def\Bernoulli{\textsf{Bernoulli}}
\def\Beta{\textsf{Beta}}
\def\Binomial{\textsf{Binomial}}
\def\Exponential{\textsf{Exponential}}
\def\LogNormal{\textsf{LogNormal}}
\def\N{\textsf{N}}
\def\Poisson{\textsf{Poisson}}
\def\Uniform{\textsf{Uniform}}
\def\WN{\textsf{WN}}
\def\GWN{\textsf{GWN}}

% Linear algebra.
\def\Col{\textsf{Col}}
\def\Dim{\textsf{Dim}}
\def\Re{\textsf{Re}}
\def\Im{\textsf{Im}}
\def\Ker{\textsf{Ker}}
\def\Null{\textsf{Null}}
\def\Rank{\textsf{Rank}}
\def\Row{\textsf{Row}}
\def\Span{\textsf{Span}}
\def\Det{\textsf{Det}}
\def\Tr{\textsf{Tr}}
\def\Diag{\textsf{Diag}}
\def\sign{\textsf{sign}}
\def\argmin{\textsf{argmin}}
\def\argmax{\textsf{argmax}}
\def\mod{\textsf{ mod }}
\def\min{\textsf{min}}
\def\max{\textsf{max}}
\def\avg{\textsf{avg}}

% Derivative at a point
% https://tex.stackexchange.com/questions/64176/derivative-at-a-point
\newcommand{\at}[2][]{#1|_{#2}}

% Logic operators.
\def\NOT{\textsf{\begin{footnotesize}NOT\end{footnotesize}}}
\def\OR{\textsf{\begin{footnotesize} OR \end{footnotesize}}}
\def\AND{\textsf{\begin{footnotesize} AND \end{footnotesize}}}
\def\XOR{\textsf{\begin{footnotesize} XOR \end{footnotesize}}}
\def\IFF{\textsf{\begin{footnotesize} IFF \end{footnotesize}}}

% For some reasons new commands can't end with a number.
%\newcommand{\v0}{\vv{0}}
%\newcommand{\v1}{\vv{1}}


% Use this if using `\contrib[]{...}`.
%\makeatletter\let\@wraptoccontribs\wraptoccontribs\makeatother
% https://tex.stackexchange.com/questions/418547/equal-contribution-using-thanks-with-llncs-class#418563
\makeatletter
\newcommand{\printfnsymbol}[1]{%
\textsuperscript{\@fnsymbol{#1}}%
}
\makeatother

\begin{document}
  \title{Artificial Intelligence-based Reasoning in Science}

  \author{Giacinto Paolo Saggese$^{*}$}
  \author{Paul Smith$^{*}$}
  %\author{Soheil Feizi$^{*}$}
  \author{Whoever else helped$^{*}$}
  \thanks{$^{*}$ Authors have contributed in equal manner and are listed in
  order of chronological contribution to this research.}

  \date{\today}

  \begin{abstract}
    This is draft v0.1.
  \end{abstract}

  \maketitle

  \setcounter{tocdepth}{2}
  \tableofcontents

  % ###############################################################################
  \section{Introduction}

  This paper proposes a novel approach to use artificial intelligence to reason
  and forecast variables in scientific disciplines, from formal sciences (e.g.,
  decision theory, statistics) to natural sciences (e.g., physics, biological,
  chemical), to social sciences (e.g., economics and finance, political
  science, business management, psychology).

  While much research has been conducted on automated reasoning on a subset of
  formal sciences (e.g., mathematics, logic) stemming from a-priori knowledge,
  very little research effort has been dedicated to automated reasoning on
  empirical and experimental sciences, where a probabilistic approach is
  essential. 

  Specifically we posit that not enough research has been dedicated to
  automated reasoning in science different to strictly formal sciences, such as
  theorem proving, which use a priori approach, as opposted to empirical
  sciences. We posit that strictly formal sciences, althugh associated with
  the human desire and search for understanding immutable truths, are not
  as concrete applicability as empirical sciences.

  We believe that the latter type of sciences are related to more
  concrete advance of the human spiecies.

  E.g., social sciences are devoted to the study of societies and the
  relationships among individuals within those societies, encompassing a wide
  array of disciplines, including economics, management and business science,
  psychology and political sciences. These disciplines have suffered from
  "physics envy" (TODO(ref)) and at the same time suffer from a cronic
  reproducibility crisis.

  % TODO(gp):
  % Forecasting and interpretability are at odds
  % Science needs ability to forecast to implement the scientific approach

  This work combines several emerging technologies in a novel way:
  \begin{enumerate}
    \item Deep-learning based AI, in particular large language models (LLMs)

    \item Knowledge graphs (KGs)

    \item Bayesian inference for probabilistic reasoning

    \item High-performance time-series streaming computing
  \end{enumerate}

  The proposed approach allows to deal with most of the problems plaguing
  forecasting in empirical and probabilistic sciences such as:
  \begin{itemize}
    \item Small and noisy datasets

    \item Need for interpretability of forecasts

    \item Need for estimation of confidence in prediction

    \item Complex, time-varying relationships between variables
  \end{itemize}

  The above problems plague current state-of-the-art AI
  \begin{itemize}
    \item Small and noisy datasets are not suitable for deep learning due to their
      tendency of overfitting when data is not sufficient

    \item Need for interpretability of forecasts: deep learning is often black box
      and lack how to interpret the value of the large number of learned weights

    \item Need for measuring confidence in prediction: LLM hallucinations related
      to the difficulty in distinguishing correct from false but plausible output,
      which can be mitigated by assigning a measure of confidence in the output

    \item Complex and time-varying models: this requires strong priors and causality
      in order to regularize learned models and make them interpretable.
      Current AI has been focusing on correlation, ignoring causation.
  \end{itemize}

  % ###############################################################################
  \section{A possible definition of AGI in science}

  We simply define Artificial General Intelligence (AGI) as a type of
  artificial intelligence that possesses the ability to understand, learn, and
  apply knowledge across a wide range of tasks at a level comparable to human
  cognitive abilities. Once this level of reasoning and intelligence is
  achieved, we believe it is possible to surpass human capabilities by
  leveraging additional data and computing power.

  % Revise the following.

  Our approach to developing AGI focuses on empirical 
  (such as economics), reflects a strategic pathway towards achieving
  broader artificial general intelligence capabilities. Here's an analysis of
  your approach and its implications:

  \begin{itemize}
    \item \textbf{Domain-Specific Expertise as a Foundation for AGI}: Focusing on
      branches of behavioral sciences initially leverages the concept that
      substantial domain-specific knowledge is essential for creating
      super-human performance. E.g., finance, with its complexity and data-rich
      environment, provides a fertile ground for developing and refining AI
      algorithms. This sector demands precision, adaptability, and the ability
      to process vast amounts of data – attributes that are foundational for
      AGI.

    \item \textbf{Stepping Stone Approach}: By mastering one domain, you
      establish a methodology and technological base that can be adapted and
      expanded to other domains. This step-by-step approach is pragmatic,
      reducing the initial complexity that would come with attempting to create
      a multi-domain AGI from the start. Each domain mastered adds another layer
      of complexity and capability to the AGI, gradually moving towards true
      general intelligence.

    \item \textbf{Integration of Diverse Data Sources and Techniques}: The ability
      to integrate and analyze data from various sources using advanced modeling
      techniques is crucial. This includes real-time processing of global market
      data, news, economic indicators, and even sociopolitical events. Such
      integrative capacity is essential for AGI, as it mirrors the multifaceted nature
      of human intelligence.

    \item \textbf{Addressing the Certainty and Hallucination Problem}: In AI,
      hallucination refers to generating or inferring information that isn't
      supported by data. An essential feature of our AGI system would be its
      ability to estimate and report the level of certainty in its predictions
      and decisions. This is a significant challenge in AI development, as it
      requires the system to not only analyze data but also understand and
      communicate the limits of its knowledge and the confidence in its
      conclusions.

    \item \textbf{Ethical and Practical Implications}: As your AGI system advances,
      it's important to consider ethical implications, particularly in
      sensitive areas like economics, finance, and other behavioral sciences.
      Issues such as data privacy, security, and the potential impact on
      employment and economic systems are vital considerations. Additionally,
      the system's decisions and predictions should be transparent and
      explainable, especially when they significantly impact financial markets
      or individual investments.
  \end{itemize}

  Questions should be answered estimating a measure of certainty (this is related
  to the problem of hallucination)

  The goal is a system that can automatically answer questions with accuracy
  superior to humans like:
  \begin{itemize}
    \item Predict the price of BTC in 7 minutes using fundamental info and news
    \item Predict the volatility of oil price in one month given the current political situation
    \item Predict the likelihood of an house crisis using only data available strictly before 2007
    \item Predict the unemployment rate this month?
    \item Predict how many Oscars “Dune 2” will win in 2024
    \item Predict the probability of a foldable IPhone by end of 2024?
    \item Predict the prescription increase this quarter of Ozempic
    \item Predict the top artist on Spotify this year
    \item Predict the highest temperature in Austin today
    \item Predict the fraud attitude of player XYZ given his betting history
    \item Estimate the effect of increasing 10\% my marketing spending on the
      demand for my product XYZ
    \item What would be the global economic impact of a sudden 2\% increase in
      the U.S. Federal Reserve interest rate today?
    \item What are the predicted movements of major currencies in the forex market
      over the next quarter, based on current geopolitical situations, trade relations,
      and economic policies?

    \item What will the rate of inflation in the next year in Italy? The current
      answer from ChatGPT is: 

      ```Predicting the exact rate of inflation for the next year in any
      country, including Italy, is a complex task that involves analyzing a
      multitude of economic indicators, current monetary policies, global
      economic trends, and unforeseen events. As of my last update in April
      2023, I don't have real-time data or the ability to predict future
      economic conditions.```

    \item Can you predict the risk of a specific stock, such as Apple, in the next
      15 minutes based on real-time market data, recent company news, and advanced
      predictive analytics?

      The current answer from ChatGPT is: 
      ```You are an expert economist. Can you predict the risk of a specific
      stock, such as Apple, in the next 15 minutes based on real-time market
      data, recent company news, and advanced predictive analytics? Answer in
      30 words```

      ChatGPT: ```As an AI, I don't have real-time market data or the
      ability to predict short-term stock movements. Accurate 15-minute
      forecasts require live data and are highly speculative due to market
      volatility.```
  \end{itemize}

  % ================================================================================
  \subsection{The need for incorporating certainty}

  Implementing a measure of certainty in AGI responses, particularly in a complex
  field like finance, is therefore not just a technical feature, but a fundamental
  aspect that enhances the utility, reliability, and ethical use of the system.

  Incorporating a measure of certainty in the responses of an AGI, especially in
  the context of finance, is crucial for several reasons:

  \begin{itemize}
    \item \textbf{Reducing the Risk of Hallucination}: AI systems, including
      AGIs, can sometimes "hallucinate", meaning they generate responses or
      predictions that are not grounded in the data they have processed. By quantifying
      the certainty of its answers, an AGI can indicate the reliability of its
      predictions, helping users differentiate between high-confidence insights and
      those that are more speculative.

    \item \textbf{Enhancing Decision-Making}: In finance, decisions often hinge
      on the level of risk and uncertainty. An AGI that can estimate and communicate
      the degree of certainty in its analysis provides invaluable information for
      risk assessment. This allows users to make more informed decisions, weighing
      the potential risks and rewards more accurately.

    \item \textbf{Building Trust}: A system that acknowledges the limits of its knowledge
      and provides certainty estimates can build greater trust with its users.
      In fields like finance, where decisions can have significant consequences,
      trust in the system's output is essential.

    \item \textbf{Dynamic Learning and Improvement}: By quantifying certainty,
      the AGI can also identify areas where its models may need improvement. Lower
      certainty in certain types of predictions can signal the need for additional
      data, refinement of models, or reevaluation of the algorithms used.

    \item \textbf{Managing Complex Systems}: Finance is a complex, dynamic
      system influenced by a multitude of factors. Providing a measure of certainty
      helps in understanding the impact of various elements and their interplay,
      acknowledging that some aspects of the financial world are inherently
      unpredictable.

    \item \textbf{Ethical and Responsible AI}: This approach aligns with ethical
      AI practices, where transparency and accountability are key. Users should be
      aware of both the capabilities and limitations of the AGI, and a measure
      of certainty facilitates this understanding.
  \end{itemize}

  % ================================================================================
  \subsection{Why focusing on behavioral sciences?}

  The application of AGI in behavioral sciences is not only a test of its predictive
  capabilities but also an exploration into understanding complex, human-centric
  systems. The success in these fields can have far-reaching implications, both
  in terms of technological advancement and societal benefits.

  The application 

  Building Artificial General Intelligence (AGI) in the field of behavioral sciences
  economics is meaningful for several reasons:

  \begin{itemize}
    \item \textbf{Limited Success of Scientific Method}: The scientific method
      faces challenges in behavioral sciences compared to natural sciences
      because human behavior and social phenomena are complex, variable, and
      influenced by cultural, historical, and subjective factors that are
      difficult to control and predict. While natural sciences can often rely
      on controlled, repeatable experiments to test hypotheses, the behavioral
      sciences deal with phenomena that cannot always be replicated in
      controlled environments, leading to a reliance on observational,
      qualitative, and interpretive research methods that may not fit the
      strict empirical mold of the scientific method.

    \item \textbf{Nature of Findings}: The findings in behavioral sciences are
      often context-dependent and may not be universally applicable due to the
      diversity of human experience and the influence of culture, history, and
      environment. These disciplines typically deal with phenomena that are
      more subjective and interpretive.

    \item \textbf{Understanding Complex Systems}: Finance and economics are intricate
      fields characterized by non-linear, interdependent variables. AGI systems could
      offer advanced understanding of these complexities, allowing for better
      predictions and insights. This is particularly valuable as financial systems
      are influenced by a myriad of factors including human behavior, market
      trends, political climates, and global events.

    \item \textbf{Dynamic Prediction Capabilities}: AGI in finance isn't just about
      static analysis but involves dynamic prediction in real-time. Financial
      markets are constantly evolving, and an AGI system can adapt to these changes,
      offering predictions and analyses that reflect current realities. This dynamic
      environment is a robust testing ground for AGI capabilities.

    \item \textbf{Managing Uncertainty and Noise}: Financial data is often noisy
      and uncertain. An AGI system's ability to sift through this noise and make
      accurate predictions despite uncertainty is a testament to its
      understanding of complex, real-world environments. This capability could
      significantly reduce risks and improve decision-making processes.

    \item \textbf{Immediate Practical Benefits}: Enhancements in economic and financial
      decision-making directly translate to benefits for the human race. Better
      financial predictions and economic models can lead to more stable
      economies, improved allocation of resources, reduced risks of financial
      crises, and overall economic growth. This can improve living standards and
      contribute to societal well-being.

    \item \textbf{Understanding Human Nature}: Since finance and economics are deeply
      intertwined with human behavior, an AGI capable of navigating these fields
      would necessarily gain insights into human psychology and behavior. This is
      crucial for building AGI that truly understands and interacts effectively with
      human-centric environments.

    \item \textbf{Ethical and Societal Implications}: Working in these domains also
      necessitates dealing with ethical considerations, such as privacy, security,
      and fairness. Successfully navigating these issues in the realm of finance
      and economics can set precedents for AGI applications in other areas,
      leading to more responsible and ethical AI development.
  \end{itemize}

  % ================================================================================
  \subsection{The problem with applying machine learning in economics and
  finance}

  Applying machine learning in economics and finance presents several challenges
  due to the unique nature of these fields, besides the traditional problems of overfitting
  (i.e., fitting the noise in the training data instead of capturing the
  underlying true dynamics, leading to poor generalization) and underfitting (i.e.,
  oversimplified models may fail to capture the complexities of financial data,
  resulting in inaccurate predictions)

  Some specific reasons that make machine learning difficult to apply in
  economics and finance are:

  \begin{enumerate}
    \item \textbf{Data Quality and Availability}:
      \begin{itemize}
        \item \textbf{Noise in Data}: Economic and financial data are often noisy
          and non-stationary. Market sentiments, geopolitical events, and
          economic policies can rapidly change, affecting the quality and relevance
          of data.

        \item \textbf{Limited Access}: High-quality, granular data can be
          expensive or restricted, limiting the scope of analysis. Proprietary
          data sources and privacy concerns add to this challenge.
      \end{itemize}

    \item \textbf{Complexity of Financial Markets}:
      \begin{itemize}
        \item \textbf{Non-Linear Relationships}: Financial markets are influenced
          by a complex web of interrelated factors. The relationships between
          these factors are often non-linear and can change over time.

        \item \textbf{Market Efficiency}: Efficient Market Hypothesis suggests that
          current asset prices reflect all available information. Predicting
          future movements based on historical data can be challenging as past
          patterns might not predict future movements.

        \item \textbf{Regime Shifts}: Economic conditions and policies can change,
          leading to regime shifts. Models trained on data from one economic regime
          might perform poorly in another.

        \item \textbf{Black Swan Events}: Unpredictable events (like the 2008
          financial crisis or the COVID-19 pandemic) can dramatically shift economic
          trends, rendering existing models ineffective.

        \item \textbf{Behavioral Factors}: Economic and financial decisions are often
          influenced by human behavior, which can be irrational and hard to
          predict. Modeling these behaviors accurately is a significant challenge.

        \item \textbf{Non-stationarity}: non-stationarity refers to data whose statistical
          properties, (e.g., mean and variance), change over time, making it
          challenging to model and predict due to evolving trends, cycles, and
          unexpected events like market crashes or economic booms. \textbf{Non-Gaussianity}:
          non-Gaussianity of random variable distributions reflects irregular,
          often extreme events, like market crashes, which are not well-described
          by the normal distribution's bell curve, leading to heavier tails and more
          pronounced risks than Gaussian models would suggest.
      \end{itemize}

    \item \textbf{Regulatory and Ethical Considerations}:
      \begin{itemize}
        \item Financial models are subject to regulatory scrutiny. Machine
          learning models, often seen as `black boxes', can raise concerns about
          transparency, accountability, and fairness.

        \item Ethical concerns, such as the risk of amplifying biases in
          financial decision-making, are also significant.
      \end{itemize}

    \item \textbf{Time Series Analysis Challenges}: Economic and financial data are
      typically time-series, which have their own challenges like
      autocorrelation, trend/cycle extraction, and handling of non-stationary
      data.

    \item \textbf{Generalization and Scalability}: Models trained on data from specific
      markets or periods may not generalize well across different contexts or times.

    \item \textbf{Feedback Loops}: Actions based on model predictions can influence
      the market, creating a feedback loop that can invalidate the model's
      assumptions.

    \item \textbf{Model Interpretability}: There's often a trade-off between model
      complexity and interpretability. In finance and economics, understanding the
      `why' behind a prediction can be as important as the prediction itself.
  \end{enumerate}

  % ###############################################################################
  \section{Large Language Models}

  % ###############################################################################
  \section{Knowledge Graphs}

  Knowledge graphs (KGs) store structured knowledge as a collection of triples
  \[
    KG = \{(h, r, t) \subset \epsilon \times R \times E\}
  \]
  where
  - $E$ is a set of entities
  - $R$ is a set of relations

  Domain-specific KGs are constructued to represent knowledge in a specific
  domain (e.g., medical, biology, and finance)

  % TODO(gp): Describe

  % ###############################################################################
  \section{Merging large Language models and knowledge graphs}

  \subsection{Pros and cons of LLMs}

  Large language models, pre-trained on large-scale corpora, have shown great
  performance in many natural language processing (NLP) tasks. By increasing model
  size, LLMs with billions of parameters (e.g., ChatGPT and PaLM2) have shown a
  surprising emergent ability and generalizability.

  Despite their success in many applications, LLMs have been criticized for several
  limitations.

  Advantages of LLMs

  \begin{itemize}
    \item General knowledge

    \item Language processing

    \item Generalizability
  \end{itemize}

  Limitations of LLMs

  \begin{itemize}
    \item Black-box: lack of interpretability

    \item Implicit knowledge: LLMs are black-box models, which fall short of capturing
      and accessing factual knowledge

    \item Hallucination

    \item Indecisiveness (reasoning happens through probabilistic process)

    \item Lack of domain-specific knowledge

    \item Lack of new knowledge
  \end{itemize}

  \subsection{Pros and cons of KGs}

  A potential solution is to incorporate knowledge graphs (KGs) into LLMs.

  Some advantages of KGs are:

  \begin{itemize}
    \item Structural knowledge

    \item Accuracy

    \item Decisiveness

    \item Interpretability

    \item Domain-specific knowledge

    \item Evolving knowledge
  \end{itemize}

  Limitations of KGs include:

  \begin{itemize}
    \item Incomplete knowledge

    \item Lack language understanding

    \item Unseen facts
  \end{itemize}

  LLMs and KGs are two inherently complementary techniques, which should be unified
  into a general framework to mutually enhance each other.

  % ###############################################################################
  \section{Prediction model}

  \subsection{World model}

  \subsubsection{Agent}

  \subsubsection{Partially observable environment}
  We assume that agent's sensors are not able to give information about the
  entire environment, but only about part of it.
  This can be due to noisy or inaccurate sensors, or to some parts of the state
  are missing from the sensor data

  \subsubsection{Belief state}
  Since the world is partially but not fully observable we need to maintain an
  internal state keeping track of the world.
  We need to model the world states (e.g., by enumeration or with formulas) and
  maintain a belief state on how likely each possible state of the world is,
  and we use probability theory to quantify the degree of belief.

  \subsubsection{Action}
  We assume that our system doesn't have ways to act on the world, but only it
  needs to estimate the state of the world and predict the next ones.

  \subsubsection{Time slices}
  We view the evolution of the world as a series of time slices (i.e., "snapshots").
  For simplicity, we assume that the length of the intervals are the same.
  Each time slice contains a set of random variables: some random variables are
  not observable (unknown, hidden), e.g., the state of the world $\vX_t$, and
  other random variables are observable, and are called evidence $\vE_t$

  Note that uncertainty over continuous time can be modeled by stochastic
  differential equation (SDEs). Discrete time models are discrete
  approximations to SDEs and so the discreteness assumption is not limiting.

  \subsubsection{Handling of time}
  TODO
  Time is handled making each quantity function of time.

  We assume that the interval between slices is fixed, so we can label times
  with integers starting from $t = 0$. Evidence starts arriving at $t = 1$
  $\vX_{[a:b]}$ represents a set of variables in $[a, b]$
  - E.g., $U_{[1:3]}$ corresponds to the variables $U_1, U_2, U_3$

  \subsubsection{Transition model}
  TODO
  Use Belief state + transition model Predict how the world might evolve in the next step

  The transition model specifies the probability distribution of the next state of
  the world $\vX_t$, given all the previous
  values:
  $$
  \Pr(\vX_t | \vX_{0:t-1})
  $$

  \subsubsection{Markov assumption for transition model}
  In general the current state $\vX_t$ depends from a growing number of past
  states:
  $$
  \Pr(\vX_t | history) = \Pr(\vX_t | \vX_0, \vX_1, ..., \vX_{t-1})
  $$
  The Markov assumption is that the current state depends only on a finite fixed
  number of previous $k$ states:
  $$
  \Pr(\vX_t | history) = \Pr(\vX_t | \vX_{t-1:t-k})
  $$
  For instance in a first-order Markov process the current state depends only
  on the previous state, and not on any earlier states
  $$
  \Pr(\vX_t | history) = \Pr(\vX_t | \vX_{t-1})
  $$
  so that a state provides enough information to make the future conditionally
  independent of the past.
  Sometimes the Markov assumption is exactly true (e.g., in the case of a
  random walk), other times it is a good approximation, depending on the
  domain.

  \subsubsection{Time invariance for transition model}
  In general the transition model depends on the value of $t$, requiring to
  specify a different distribution for each time step.
  We assume that the changes in the state of the world are time-homogeneous,
  i.e., the laws that govern the world don't change over time
  E.g., this assumption combined with a first-order Markov process yields
  $$
  \Pr(\vX_t | history) = \Pr(\vX_t | \vX_{t-1}) = f(X_{t-1}) \forall t
  $$

  \subsubsection{Sensor model}
  In general the model to measure the evidence holds
  $$
  \Pr(\vE_t | \vX_{[0:t]}, \vE_{[1:t-1]})
  $$
  Under mild hypothesis, the model can be assumed to follow a form:
  $$
  \Pr(\vE_t | \vX_t)
  $$

  \subsubsection{Prior probability distribution at time 0}
  The world model requires to specify the initial conditions $\Pr(\vX_0)$.

  \subsubsection{Joint distribution}
  Given a transition model following a first-order Markov process and sensor
  model, and the prior distribution we can infer the joint distribution overall
  all the variables:
  $$
  \Pr(\vX_{[0:t]}, \vE_{[1:t]}) = \Pr(\vX_0) \prod_{i=1} \Pr(\vX_i | \vX_{i-1}) \Pr(\vE_i | \vX_{i-1})
  $$

  \subsubsection{Update belief state}
  TODO
  Use sensor model + percepts to update belief state

  % TODO(gp): Describe

  % ###############################################################################
  \section{Description of the architecture}

  \subsection{Offline pass}

  \begin{itemize}
    \item Step 1: generate domain-specific knowledge graph

      \begin{itemize}
        \item The knowledge graph contains informations like:

          \begin{itemize}
            \item Type of dependencies (e.g., causal, correlation)

            \item Formula representing the type of relationship (e.g., sign, linear)

            \item Strength of the relationship (in terms of belief)
          \end{itemize}

        \item We developed automatic ways (e.g., using LLMs) to generate a
          knowledge graphs of the domain

        \item Human-in-the-loop approach to ``regularize'', inspect, and
          improve it
      \end{itemize}
  \end{itemize}

  \subsection{Online pass}

  \begin{itemize}

    \item Step 1: User poses a question to system

    \item Step 2: An LLM parses the question and extracts a relevant subset
      of the KG associated to the topic of the question

    \item Step 3: The subset of KG is converted into a BN that will answer
      the question

    \item Step 4: Relevant data (e.g., time series) needed to evaluate the
      BN is retrieved

    \item Step 5: The BN is evaluated, e.g., using KaizenFlow, in terms of
      values and uncertainties

    \item Step 6: The answer to the question is returned together with
      estimates about the error and the epistemological conviction
  \end{itemize}

  % ###############################################################################
  \section{Building Bayesian model}

  % TODO(gp): Describe

  % ###############################################################################
  \section{Running the model}

  % TODO(gp): Describe

  % ###############################################################################
  \bibliography{Bib}
  \bibliographystyle{amsplain}

  \begin{thebibliography}{10}
    \bib{TrDeProv23}{article}{ author={Trace}, title={Decentralized Proving, Proof Markets, and ZK Infrastructure}, date={June 2023}, eprint={https://figmentcapital.medium.com/decentralized-proving-proof-markets-and-zk-infrastructure-f4cce2c58596} }

    %    \bib{}{article}{
    %      author={,},
    %      title={},
    %      date={},
    %    }
  \end{thebibliography}
\end{document}
